%%
%% This is a LaTeX template file for EPS
%%
%% modified by M. M. on 17 March 2014
%% modified by Y.O. on 7 September 2014

% \documentclass[submit]{eps}
\documentclass{EPS}

\usepackage{lineno}
\linenumbers*[1]

\newcommand{\bvec}[1]{\mbox{\boldmath $#1$}}

\setcounter{page}{1}
\title{Three-dimensional structure of the boundary current in a mini-magnetosphere above a lunar magnetic anomaly 
}
\author{
	Hideyuki Usui, 
	Graduate school of system informatics, Kobe University, 
	1-1 Rokkodai-cho, Nada-ku, Kobe 657-8501, Japan, h-usui@port.kobe-u.ac.jp ¥¥
	Takuma Matsubara, 
	Graduate school of system informatics, Kobe University, 
	1-1 Rokkodai-cho, Nada-ku, Kobe 657-8501, Japan, h-usui@port.kobe-u.ac.jp ¥¥
	Masaki Nishino, 
	Solar-Terrestrial Environment Laboratory, Nagoya university, 
	Furo-cho, Chikusa-ku, Nagoya, 464-8601, Japan, mnishino@stelab.nagoya-u.ac.jp  ¥¥
	Yohei Miyake, 
	Graduate school of system informatics, Kobe University, 
	1-1 Rokkodai-cho, Nada-ku, Kobe 657-8501, Japan, y-miyake@eagle.kobe-u.ac.jp ¥¥
	Misako Umezawa, 
	Graduate school of system informatics, Kobe University, 
	1-1 Rokkodai-cho, Nada-ku, Kobe 657-8501, Japan, h-usui@port.kobe-u.ac.jp
}

\abstract{
We studied the three-dimensional structure of the boundary current in a mini-magnetosphere created 
above the magnetic anomaly on the lunar surface by performing electromagnetic particle-in-cell simulations.
The size of a magnetic anomaly can be characterized by the distance $L$ from the center of the magnetic dipole 
to the position where the pressure of the local magnetic field becomes equal to the dynamic pressure of 
the solar wind under the magnetohydrodynamics (MHD) approximation. 
In this study, we focused on a magnetic anomaly which has $L$ smaller than the local ion Larmor radius 
$r_\mathrm{iL}$
observed at the distance of $L$ from the dipole center. 
In the simulation results for the case $r_\mathrm{iL} /L=4$, we confirmed that a mini-magnetosphere is 
formed by the interaction of the magnetic anomaly with the incoming solar wind plasma. 
As examined in the previous studies, we also found an asymmetric density profile of the mini-magnetosphere 
with respect to the solar wind direction. 
At the boundary layer of the magnetosphere, intense boundary current is observed. 
We found that the boundary current is mainly due to cross-field motions of electrons 
whose Larmor radius is much smaller than $L$. 
Considering that intense outward electric field is induced by the charge separation between 
the magnetized electrons and the weakly magnetized ions at the boundary layer, 
the solar wind electrons encountering the magnetic anomaly region make the $E ¥times B$ drift motion 
in perpendicular to the magnetic dipole. 
In the case where the direction of the magnetic field is southward in the dayside region, 
the electron drift motion is in the dawn-to-dusk direction in the low to mid-latitude region. 
At the high latitude region, on the other hand, the direction of the electron cross-field motion changes and 
it becomes from the dusk to the dawn side. 
It is because of the curvature of the dipole field and the direction of the magnetic field 
lines becomes locally different in the high latitude region from that in the equatorial plane 
while the induced electric field basically points outward at the boundary layer. 
We revealed that this two-layer structure of electron flow between the low and mid-latitude 
and the high latitude is very characteristic in the mini-magnetosphere created above magnetic anomaly.

}


\keywords{magnetic anomaly, lunar environment, PIC simulation, mini-magnetosphere}

\begin{document}

\maketitle

\section{Introduction}
The lunar plasma environment has been intensively investigated with the recent in situ observation 
by scientific spacecraft such as KAGUYA[参考文献] and Chandrayaan-1[参考文献]. 
In addition to the macroscopic structure of plasma distribution such as the wake structure 
with low density of plasma formed in the downstream region, small-scale perturbation of 
plasma distribution and fields are newly observed in the dayside region mostly 
over the crustal magnetic anomalies found on the lunar surface. 
In the KAGUYA spacecraft observations, more than $10 \%$ of solar wind protons are reflected 
at 100km over the magnetic anomalies [Saito et al., 2010]. 
Chandrayaan-1 found the deflection of the solar wind protons with high efficiency more than $ 10 \%$ [Lue et al., 2011]  
and it also observed the backstreaming ions over the magnetic anomalies [Wieser et al., 2010]. 
In association with the plasma variation, plasma wave activities are also enhanced over 
the magnetic anomalies [???].
To understand these solar wind responses, double layer structure at the dayside magnetopause was 
proposed from the observational point of view [Lue et al., 2011], [Saito et al., 2012]. 
These observational facts suggest that the plasma and field disturbances over 
the magnetic anomalies are caused by the solar wind interactions with the local magnetic fields of the magnetic anomalies. 
Unlike the case of the Earth’s magnetosphere, however, the dipole moment of 
the magnetic anomalies on the lunar surface is very weak and the resulting dipole field region 
is much smaller than the characteristic spatial scale of the solar wind such as 
the ion inertial length and the ion's Larmor radius $r_\mathrm{iL}$. 
When we define $L$ as the typical scale of a magnetic dipole, 
$L$ can be an equilibrium point with the MHD approximation where the plasma dynamic pressure balances 
to the local magnetic pressure of the dipole field. 
In the situation of the lunar magnetic anomalies, $L$ can be less than 100km and 
becomes smaller than $r_\mathrm{iL}$ of the solar wind proton. 
In such a situation, the protons are assumed unmagnetized and the interactions with the dipole field becomes 
loose because the finite Larmor radius effect cannot be ignored. 
Then the ions dynamics can be little affected by the local field of magnetic anomalies. However, as stated above, 
variation of ion dynamics such as the ion reflection are obviously observed by spacecraft. 
Plasma flow response to such a small-scale magnetic dipole has been examined in different situations. 
When $L$ is comparable to or smaller than $r_\mathrm{iL}$ 
but large enough in comparison with the electron Lamor radius $r_\mathrm{eL}$ 
we call this range meso-scale. 
As one of the feasibility studies of a future interplanetary flight system called magneto-sail, 
the solar wind interactions with a meso-scale magnetic dipole artificially created by a 
superconducting cold equipped at a spacecraft has been examined with two-dimensional 
particle-in-cell plasma simulations [Moritaka et al., 2012]. 
They confirmed a clear formation of a magnetosphere even in a meso-scale regime. 
In the interactions between the solar wind and a meso-scale magnetic dipole, 
it can be assumed that electrons are magnetized while ions are not much at the magnetopause. 
Then electric field is induced at the dayside magnetosphere by charge separation due to 
the momentum difference between electrons and ions. 
The plasma responses to the induced electric field were also examined 
in the previous simulation studies using particle model 
[Harnett and Winglee, 2002; Kallio et al., 2012; Poppe et al., 2012; Deca et al., 2015]. 
Laboratory experiments associated with the plasma interactions with small-scale magnetic dipole 
were also conducted and electric potential and electric field in the interactions were examined 
[Wang et al., 2013] [Bamford et al., 2012] [??????, boulder 2015] .
 As pointed in many previous studies, plasma kinetics should be included in modeling the interactions 
 between the solar wind and the magnetic anomalies because  $L$ is smaller than
$r_\mathrm{iL}$. 
 In this letter,  by performing three-dimensional electromagnetic particle-in-cell simulations we will 
 describe the spatial structure of the boundary current layer in the mini-magnetosphere formed 
 above a magnetic anomaly. 
 The boundary current in the mini-magnetosphere mainly consists of electron current. 
 Therefore we particularly focus on the electron dynamics in the boundary layer 
 in the three-dimensional simulation results. 
  In Section II, we describe the simulation model used in this study. 
 In our model we mainly examined a case of $r_\mathrm{iL} /L=4$. 
for the magnetic anomaly. 
 In Section III, we describe the spatial variation of plasma density and 
 the current density obtained in the simulations. 
We also examine the electron dynamics in the boundary layer in the three-dimensional domain. 
 In Section IV, we summarize the results.  




For citing a reference, please follow
``Basic Springer Reference Style''.
For example,
\cite{Miller2009}, \cite{MS2001}, \cite{Milleretal1999},
which are for papers with one, two and more than two authors.
You may also write the citations like this
\citep[e.g.][]{Miller1998a,Miller1998b,MS2001,Milleretal1999}.


\section{Methods}


We performed three-dimensional, full kinetic, electromagnetic PIC simulations to 
examine the boundary layer of a mini-magnetosphere above a magnetic anomaly. 
In the simulation, Reiner Gamma, which is one of the typical magnetic anomalies on the lunar surface, 
is refereed to as a model of magnetic anomaly. 
In the Reiner Gamma case, the magnetic dipole moment is almost parallel to the lunar surface. 
The dipole center is approximately located at a depth of 10 km under the ground. 
The amount of the dipole moment is ***.
Figure \ref{fig:1} shows a three-dimensional simulation model. In the positive x direction, 
we have a plasma flow representing the solar wind. 
The static magnetic field representing IMF is also introduced in the simulation domain. 
In this study, the direction of IMF is northward. 
The Mach number $M$ which is a ratio of the plasma flow velocity $V_\mathrm{flow}$ to the Alfven velocity $V_\mathrm{A}$ is 5. 
The values used for the plasma flow parameters are as follows. 
Instead of using actual measured physical values, we used scaled ones. 
To reduce the calculation time, we set the mass ratio to be a value between the mass of an ion and that of an electron, 
$ m_\mathrm{i}/ m_\mathrm{e}$ = 100 (これ梅沢モデルだよね。。). 
We set the ratio of the speed of light $V_\mathrm{c}$ to $ V_\mathrm{flow}$ to be 25, although the actual ratio is about 600. 
We set the ratio of the electron thermal velocity $V_\mathrm{th}$ to $V_\mathrm{flow}$ to be 2.5, 
which is almost the same as the actual ratio for the solar wind. 
When $ r_\mathrm{iL} /L = 4$, the ratio of the electron plasma frequency to the cyclotron frequency, 
$\omega_\mathrm{pe} / \Omega_\mathrm{e} = 4$, was set to be 250 at a distance L from the dipole center. 
The simulation domain consists of $256 \times 256 \times 256$ cells, and the length of each side of 
the domain was approximately $5L$. 
The cell size dx corresponds to $\lambda_\mathrm{D}$, where $\lambda_\mathrm{D}$ 
denotes the Debye length in the solar wind plasma.
The time increment dt was chosen to satisfy the Courant condition, $dx/dt > 1.73 V_c $, in three dimensions. 
The plasma flow typically consists of $4 \times 10^9 $ macroparticles in the simulation space. 
In the center of the simulation domain, we set an ideal dipole magnetic field with a magnetic moment $M$. 
The magnetic moment is arranged in the $z$ direction, and the resulting magnetic field was taken into account in the equation of motion when the velocity of each particle was updated. 
Absorbing boundary conditions were used for the electromagnetic field on all of the outer boundaries. 
We flow plasma particles into the simulation domain to the lunar surface at the velocity of $V_\mathrm{flow}$
with a thermalized distribution. Particles leaving the simulation domain are eliminated from the calculation. 
Since we assumed space plasma, we assumed there were no collisions between particles. 
We performed the simulation to iterate until the spatial variation of the plasma density reached a steady state. 


\section{Results and Discussion}


Figure \ref{fig:2} shows contour maps of the number densities of electrons and ions and the charge 
density obtained at the steady state in the meridian plane which includes the sunward direction. 
As shown in the left two panels, a clear magnetosphere which is a plasma evacuated region is 
created above the magnetic anomaly region as studied in the previous works such as *******. 
In panel (a) which shows the electron density, the boundary of the magnetosphere seems 
very clear along the dipole fields and the density becomes relatively high in the high latitude regions 
because of the convergence of the magnetic fields.  
In panel (b) indicating the ion density, we also recognize the magnetosphere as shown in the electron density map. 
However, the boundary does not seem so clear as in the electron case. 
Across the electron boundary region shown in panel (a), some ions reach the inner magnetosphere 
where no electrons exist. 
The magnetosphere itself seems compressed along the horizontal axis corresponding to the solar wind direction. 
The ion density is also relatively high in the high latitude regions. 
In comparison between panels (a) and (b), the spatial profiles of the magnetosphere are similar to each other. 
However, because there are some differences in the density profiles as stated above, 
we found a spatial variation in terms of charge density as shown in the contour map in panel (c). 
Since the charge density variation causes the electric potential, 
we superimpose arrays of the resulting electric field on the contour map. 
The contour map in panel (c) shows that ions are rich in the inner magnetosphere as shown in red region 
while the boundary layer surrounding the ion rich region is shown in blue indicating electrons are relatively rich. 
Because of this charge separation at the boundary layer, intense electric field is induced 
as shown in the arrows pointing radially to the outward direction to the magnetosphere.  (2015/09/11)

Figure \ref{fig:3} shows contour maps of the number densities for electrons and ions and the charge density 
in the equatorial plane in the same manner as in Figure \ref{fig:3}. 
One of the interesting features we notice is asymmetric profile of the magnetosphere. 
As shown in panels (a) and (b), a high density region is created 
at the upper side of the boundary of the magnetosphere which corresponds to the dawn side. 
Panel (c) shows the charge density. 
The profile is overall similar to that in panel (c) in Figure \ref{fig:2}. Ions which have larger inertia 
than electrons reach the inner magnetosphere and an ion rich region is created. 
As shown in panels (a) and (b), the charge density also has an asymmetric profile. 
In the dawn region, electron rich region is locally created near the lunar ground. 
Due to this charge separation in the boundary region of the magnetosphere, 
intense electric field pointing outward is also found in the equatorial plane. 
Because of the electron rich region at the dawn side near the surface, some electric fields point there. 

Figure \ref{fig:4} shows the contour maps of the total current density at the different latitudes. 
Panels (a) and (b) show the profiles in the equatorial plane and in high latitude plane of z=*** respectively. 
The vector arrays of the current components in the $x-y$ plane are superimposed on the contour maps. 
In the equatorial plane shown in panel (a), 
the current density is intense at the boundary layer of the magnetosphere. 
The direction of the current is along the positive $y$ direction which corresponds to the dawn direction. 
As shown in the density profile, the current density profile is also asymmetric and 
intense current is found at the dawn side. 
In panel (b), however, the direction of the most intense current is different from that in panel (a). 
It is observed near the lunar surface pointing to the dust side. 
To examine the detail of the current, we decompose the total current into electron current and 
ion current as shown in Figure \ref{fig:5}. 
We plot the electron and ion current densities measured along a line of $y=0$ in black and red, respectively. 
Panels (a) and (b) show the profiles in the different latitudes in the same manner as shown in the previous figure. 
As shown in the both panels, the ion current shown in red is relatively small in comparison with the electron current. 
It implies that the boundary current is dominated by the electron current. 
In panel (a), the ion current has a peak at the inner magnetosphere where almost no electrons are found. 
The current flows in the positive $y$ direction. In panel (b) corresponding to the high latitude region, 
the electron current has two peaks in the opposite directions. 
One found around x=*** point to the positive y direction while the other found around $x=**$ 
which is close to the ground has a much higher peak pointing to the negative y direction. 
These profiles agree with the vector plots shown in Figure \ref{fig:4}. 
To see the overall profile of the boundary current, 
we plot a all sky view of the electron current with the vector arrays in Figure \ref{fig:5} 
which is observed from the ground in the magnetic anomaly region. 
As shown in the figure, 
two current voltices are found in the upper and lower regions 
with respect to the equatorial plane which is located in the middle.
As stated above, 
the electron current which flows from the dawn to the dusk side in the equatorial and low latitude region
changes the flow direction and returns back to the dawn region in the high latitude region.
In other words, it seems that the electron current makes a loop in the both hemispheres. 

Figure \ref{fig:6} shows a contour map of the $y$ component of the electron current density in a meridian plane.
As discussed earlier, the electron current is dominant in the positive $y$ direction which corresponds to the
dawn direction in the low latitude region including the equatorial region.
In the high latitude region, however, the electron current is reversed to the negative $y$ direction. 
This reverse of current can cause a current loop connecing between the lower and higher latitude 
as shown in the previous figure.  

\section{Discussion}

We found that the boundary current mainly consists of electron current. 
In the electron current, cross-field component which is along the $y$ direction is dominant. 
Meanwhile $r_\mathrm{eL}$ is much smaller than $L$ in the present case. 
In this situation, the cross-field component of the current is due to the drift motion of electrons.
Here we examine the drift motion of electrons in the equatorial plane. 
As observed in Figure \ref{fig:2} and \ref{fig:3}, we have intense electric field poinging in the outward direcion
in the boundary layer region. 
Considering we have dipole magnetic field in the negtive $z$ direction in the present case, 
the direction of $E \times B$ drift motion is along the negative $y$ direction which corresponds to the 
dawn to the dusk direction.  
Figure \ref{fig:6} shows spatial variation of the $E \times B$ velocity along the ling of $y=z=0$. 
 
\begin{linenomath}
 \begin{equation}
  m_\mathrm{e} n_\mathrm{e} \frac{d \bvec{v}}{dt}
    = en_\mathrm{e}(\bvec{E} + \bvec{v} \times \bvec{B}) - \nabla P_\mathrm{e}
 \end{equation}
\end{linenomath}

When we assume the steady state, $\partial{\bvec{v}}/\partial{t} = 0 $ in the above equation. Then 
the following equation is obtained.

\begin{linenomath}
 \begin{equation}
  m_\mathrm{e} n_\mathrm{e} v_\mathrm{x} \frac{\partial v_\mathrm{x}}{\partial t}
    = en_\mathrm{e} (E_x + v_y B_z) -  K T_\mathrm{e} \frac{\partial n_\mathrm{e}} {\partial x}
 \end{equation}
\end{linenomath}
From this relation, we can obtain $v_y$ as follows.


\begin{linenomath}
 \begin{equation}
  v_{y} = 
   -\frac{E_x}{B_z} + \frac {m_\mathrm{e}}{eB_z}v_x\frac{\partial v_x}{\partial x}
   +\frac{KT_e}{en_e B_z} \frac{\partial n_\mathrm{e}} {\partial x}
 \end{equation}
\end{linenomath}

In the above equation, 
the effects of electron kinetic such as $v_x$ component as well as 
density variation $\nabla n_\mathrm{e}$ along the $x$ direction is taken into consideration. 
By using the simulation data such as $B_z$, $E_x$, $n_e$, and $v_x$ obtained at each grid point, 
we can estimate $v_y$ valuealong the line of our interest with the above equation. 
The estimated $v_y$ values are plotted with a solid line while $v_y$ values obtained in the simulation 
is plotted in a dashed line.  (wait for the Matsubara-kun's result.)


As shown in Figure \ref{fig:5} and \ref{fig:6} 
the direction of the electron current in the high latitude region 
is opposite to that in the low latitude region.
As discussed above, the electron current is maily due to the electron $\bvec{E} \times \bvec{B}$ 
drift motion. In this aspect, the direction of the electron $\bvec{E} \times \bvec{B}$ motion is 
reversed in the high latitude region.






\section{Conclusions}

This should state clearly the main conclusions of the research
and give a clear explanation of their importance and relevance.
Summary illustrations may be included.

\section{List of abbreviations}

If abbreviations are used in the text
they should be defined in the text at first use,
and a list of abbreviations can be provided,
which should preced the competing interests and authors' contributions.

\section{Competing interests}

A competing interest exists when your interpretation of data or
presentation of information may be influenced by your personal or
financial relationship with other peaple or organizations.
Authors must disclose any financial competing interests;
they should also reveal any non-financial competing interests
that my cause them embraassment were they to become public
after the publication of the manuscript.

\section{Authors' contributions}

In order to give appropriate credit to each author of a paper,
the individual contributions of authors to the manuscript
should be specified in this section.

\section{Authors' information}

You may choose to use this section to include any relevant
information about the author(s) that may aid the reader's interpretation
of the article, and understand the standpoint of the author(s).
This may include details about the authors' qualifications,
current positions they hold at institutions or societies,
or any other relevant background information.
Please refer to authors using their initials.
Note this section should not be used
to describe any competing interests.

\acknowledgments{%

Authors should obtain permission to acknowledge
from all those mentioned in the Acknowledgements again.
}

\appendix{An example of appendix}
{This is the appendix.}

\section{Endnotes}

Endnotes should be designated within the text using a
superscript lowercase letter and all notes
(along with their corresponding letter) should be included
in the Endnotes section.
Please format this section in a paragraph rather than a list.

\begin{thebibliography}{}
%+--------------------------------------------------------------------+
%| Examples of the Earth, Planets and Space reference style as below. |
%+--------------------------------------------------------------------+
% References used in Background section above
% --------------------------------------------
\bibitem[{{Miller}(1998a)}]{Miller1998a}
Miller X (1998a) Sample of references. Earth Planets Space 55:1-20

\bibitem[{{Miller}(1998b)}]{Miller1998b}
Miller X (1998b) Sample of references. Earth Planets Space 55:21-40

\bibitem[{{Miller}(2009)}]{Miller2009}
Miller X (2009) Sample of references. Earth Planets Space 66:1-20

\bibitem[{{Miller and Smith}(2001)}]{MS2001}
Miller X, Smith Z (2001) First sample of references.
Earth Planets Space 65: 22-34

\bibitem[{{Miller et al.}(1999)}]{Milleretal1999}
Miller X, Smith Z, Third A (1999) First sample of references.
Earth Planets Space 56: 12-34
% --------------------------------------------

% Article within a journal
% ------------------------
\bibitem[{{Smith et al.}(1999)}]{SJH1999}
Smith J, Jones M Jr, Houghton L (1999)
Future of health insurance.
N Engl J Med 956:325-329

% Article by DOI (with page numbers)
% ----------------------------------
\bibitem[{{Slifka and Whitton}(2000a)}]{SW2000a}
Slifka MK, Whitton JL (2000a)
Clinical implications for dysregulated cytokine production.
J Mol Med 78:74-80. doi:10.1007/s001090000086

% Article by DOI (before issue publication and with page numbers)
% ---------------------------------------------------------------
\bibitem[{{Slifka and Whitton}(2000b)}]{SW2000b}
Slifka MK, Whitton JL (2000b)
Clinical implications of dysregulated cytokine production.
J Mol Med doi:10.1007/s001090000086

% Article in electronic journal by DOI (no paginated version)
% -----------------------------------------------------------
\bibitem[{{Slifka and Whitton}(2000c)}]{SW2000c}
Slifka MK, Whitton JL (2000c)
Clinical implications of dysregulated cytokine production.
Dig J Mol Med doi:10.1007/s001090000086

% Journal issue with issue editor
% -------------------------------
\bibitem[{{Smith}(1998a)}]{Smith1998a}
Smith J (ed) (1998a)
Rodent genes. Mod Genomics J 14(6):126-233

% Journal issue with no issue editor
% ----------------------------------
\bibitem[{{Smith}(1998b)}]{Smith1998b}
Smith J (1998b)
Rodent genes. Mod Genomics J 14(6):126-233

% Book chapter, or an article within a book
% -----------------------------------------
\bibitem[{{Brown and Aaron}(2001)}]{BBAM2001}
Brown B, Aaron M (2001)
The politics of nature. In: Smith J (ed)
The rise of modern genomics, 3rd edn. Wiley, New Yourk

% Comlete book, authored
% ----------------------
\bibitem[{{Smith and Brown}(2001)}]{SB2001}
Smith J, Brown B (eds) (2001)
The demise of modern genomics. Blackwell, London

% Complete book, also showing atranslated edition
% [Either edition may be listed first.]
% ------------------------------------------------
\bibitem[{{Adrno}(1966)}]{Adrno1966}
Adrno TW (1966)
Negativ Dialektk. Suhrkamp, Frankfurt.
English Edition: Adorno TW (1973)
Negative Dialetics (trans: Ashton EB). Routledege, London

% Chapter in a book in a series without volume titles
% ---------------------------------------------------
\bibitem[{{Schmidt}(1989)}]{Schmidt1989}
Schmidt H (1989)
Testing results. In: Hutzinger O (ed)
Handbook of environmental chemistry, vol 2E.
Springer, Heidelberg, p 111

% Chapter in a book in a series with volume titles
% ------------------------------------------------
\bibitem[{{Smith}(1976)}]{Smith1976}
Smith SE (1976)
Neuromescular blocking drugs in man.
In: Zaimis E (ed) Neuromuscular junction.
Handbook of experimental pharmacology, Vol 42.
Springer, Heidelberg, pp 593-660

% OnlineFirst chapter in a series
% (without a volume designation but with a DOI)
% ---------------------------------------------
\bibitem[{{Saito and Hyuga}(2007)}]{SaitoHyuga2007}
Saito Y, Hyuga H (2007)
Rate equation approaches to amplification of enantiomeric excess
and chiral symmetry breaking.
Topics in Current Chemistry. doi:10.1007/128\_2006\_108

% Proceedings as a book (in a seris and subseries)
% ------------------------------------------------
\bibitem[{{Zowghi}(1996)}]{Zowghi1996}
Zowghi D (1996)
A framework for reasoning about requirements in evloution.
In: Foo N Goebel R (eds) PRICAI'96: topics in artificial intelligence.
4th Pacific Rim conference on artificial intelligence,
Cairns, August 1996.
Lecture notes in computer science
(Lecture notes in artificial intelligence), vol 1114.
Springer, Heidelberg, p 157

% Article within conference proceedings with an editor (without a publisher)
% --------------------------------------------------------------------------
\bibitem[{{Aaron}(1999)}]{Aaron1999}
Aaron M (1999)
The future of genomics.
In: Williams H (ed) Proceedings of the genomic researchers, Boston, 1999

% Article presented at a conference
% ---------------------------------
\bibitem[{{Chung and Morris}(1978)}]{ChungMorris1978}
Chung S-T, Morris RL (1978)
Isolation and characterzation of plasid deoxyribonucleic acid
from Steptomyces fradiae.
Ppaer presented at the 3rd international symposium
on the genetics of industrial microorganisms,
University of Wisconsin, Madison, 4-9 June 1978

% Patent
% ------
% Norman LO (1998)
% Lightning rods. US Patent 4,379,752, 9 Sept 1998

% Dissertation
% ------------
\bibitem[{{Trent}(1975)}]{Trent1975}
Trent JW (1975)
Experimental acute renal failure.
Dissertation, University of California

% Book with institutional author
% ------------------------------
\bibitem[{{International Anatomical Nomenclature Committee}(1966)}]{IANC1966}
International Anatomical Nomenclature Committee (1966)
Nomina anatomica. Excerpta Medica, Amsterdam

% In press article
% ----------------
\bibitem[{{Major}(2007)}]{Major2007}
Major M (2007)
Recent developments: In Jones W (ed) Surgery today.
Springer, Dordrecht (in press)

% Online document
% ---------------
\bibitem[{{Doe}(1999a)}]{Doe1999a}
Doe J (1999a)
Title of subordinate document.
In: The dictionary of substances and their effects.
Royal Society of Chemistry. Available via DIALOG.
http://www.rsc.org/dose/titlle of subordinate document.
Accessed 15 Jan 1999

% Online database
% ---------------
\bibitem[{{Healthwise Knowledgebase}(1998)}]{HK1998}
Healthwise Knowledgebase (1998)
US Pharmacopea, Rockville.
http://www.healthwise.org. Accessed 21 Sept 1998

% Suppleentary material/private homepage
% -------------------------------------- 
\bibitem[{{Doe}(2000)}]{Doe2000}
Doe J (2000)
Title of supplementary material.
http://www.privatehomepage.com. Accessed 22 Feb 2000

% University site
% --------------- 
\bibitem[{{Doe}(1999b)}]{Doe1999b}
Doe J (1999b)
Title of preprint.
http://www.uni-heidelberg.de/mydata.html. Accessed 25 Dec 1999

% FTP site
% --------
\bibitem[{{Doe}(1999c)}]{Doe1999c}
Doe J (1999c)
Trivial HTTP, RFC2169.
ftp://ftp.isi.edu/in-notes/rfc2169.txt. Accessed 12 Nov 1999

% Organization site
% -----------------
\bibitem[{{ISSN International Centre}(2006)}]{ISSNIC2006}
ISSN International Centre (2006)
The ISSN register. http://www.issn.org. Accessed 20 Feb 2007

\end{thebibliography}{}

\section{Preparing illustrations and figures}

Illustrations should be provided as separate files,
not embedded in the text file.
Each figure should include a single illustration and
should fit on a single page in portrait format.
If a figure consists of separate parts,
it is important that a single composite illustration file
be submitted which containes all parts of the figure.
There is no charge for the use of color figures.

Please read our figure preparation guidelines
for detailed instruction on maximising the quality of your figures.

\noindent
\textbf{Formats}

The following file formats can be accepted:
\begin{itemize}
\item PDF (preferred format for diagrams)
\item DOCX/DOC (single page only)
\item PPTX/PPT (single slide only)
\item EPS
\item PNG (preferred format for photos or images)
\item TIFF
\item JPEG
\item BMP
\end{itemize}

\noindent
\textbf{Figure legends}

The legends should be included in the main manuscript text file
at the end of the document, rather than being a part of the figure file.
For each figure, the following information should be provided:
Figure number (in sequence, using Arabic numerals -
i.e. Fiugre 1, 2, 3 etc); short title of figure (maximum 15 words);
detailed legend, ,up to 300 words.

\textbf{Please note that it is the responsibility of the author(s)
to obtain permission from the copyright holder
to reproduce figures or tables that have previously
been published elsewhere.
}

\section{Preparing tables}

Each table should be numbered and cited in sequence
using Arabic numerals (i.e. Table 1, 2, 3 etc).
Tables should also have a tilte (above the table)
that summarizes the whole table;
it should be no longer than 15 words.
Detailed legends may the follow, but they should be concise.
Tables should always be cited in text in consecutive numerical order.
Smaller tables consdered to be integral to the manuscript can be
paseted into the end of the document text file,
in A4 portrait or landscape format.
These will be typeset and displayed in the final published form
of the article.
Such tables should be formatted using the `Table object'
in a word processing program to ensure that
columns of data are kept aligned when the file is sent
electronically for review;
this will not always be the case if columns are generated by
simply using tabs to separate text.
Columns and rows of data should be made visibly distinct
by ensuring that the borders of ech cell display as black lines.
Commas should not be used to indicate numerical values.
Color and shading may not be used;
parts of the table can be highlighted using symbols or bold text,
the meaning of which should be explained in a table legend.
Tables should not be embedded as figures or speadsheet files.

Larger datasets or tables too wide fo a portrait page
can be uploaded separately as additional fiels.
Additional files will not be displayed in the final,
laid-out PDF of the article, but a link will be provided
to the files as supplied by the author.
Tabular data provided as additional files can be uploaded
as an Excel spredsheet (.xls) or comma separeted values (.csv).
As with all files, please use the standard file extensions.

%% - Figure 1-
\begin{figure}
\caption{Three-dimensional simulation model}\label{fig:1}
\end{figure}

%% - Figure 2-
\begin{figure}
\caption{Conour maps of electron and ion densities and the charge density 
obtained at the steady state in a meridian plane}\label{fig:2}
\end{figure}

%% - Figure 3-
\begin{figure}
\caption{Conour maps of electron and ion densities and the charge density 
obtained at the steady state in the equatorial plane}\label{fig:3}
\end{figure}

%% - Figure 4-
\begin{figure}
\caption{Contour maps of current densities in the equatorial plane and in a plane of $z=???$}\label{fig:4}
\end{figure}

%% - Figure 5-
\begin{figure}
\caption{Electron and ion current densities along $y=z=0$ and $y=0, z=???$ }\label{fig:5}
\end{figure}

%% - Figure 6-
\begin{figure}
\caption{All skyview map of electron current densities with arrows }\label{fig:6}
\end{figure}
\end{document}
